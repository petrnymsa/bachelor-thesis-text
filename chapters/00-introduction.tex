In these days, there are many systems, that help companies to analyse and monitor their business processes and performance. These systems are well established and widely used. Besides of monitoring, better and more complex systems offer the whole bunch of functionalities, from the defining business process, running them to analysing and monitoring. However, these systems are built on top of notation BPMN, which is again, well known and widely used. The standardly used term for these systems is Business Process Management System (BPMS).

This thesis relies on the theory of another approach - DEMO methodology, which is quite new and there are likely no systems build on this methodology. The thesis is not focused on every aspect of the existing BPMS solutions. The thesis is only focused on how processes can be monitored and displayed in graphical overviews. How processes can be modelled and executed is the different topic, which was researched for example by \textit{M. Skotnica}\cite{diploma-skotnica-2016}. 

The motivation for focusing on this problem comes out from the school project, which was about creating chat-bot for pizza delivery system. This chatbot was relying on DEMO Engine, which is the system, that can design and run DEMO models. However, this system did not have the ability to display or visualise processes in some ``user-friendly'' graphical overview. These graphical overviews can help to better understand business processes, monitor performance and many more metrics, which can lead to more effective business.

\subsection{Structure}
The thesis is organized into main four chapters:
\begin{enumerate}
\item In \cref{ch:theoretical-foundations} underlying theories are introduced and described.
\item In \cref{ch:bpms} research of existing BPMS solutions follows. Then some specific solutions are showed and how knowledge from research is used to propose a similar system based on DEMO.
\item In \cref{ch:proposed-approach} proposed approach is explained. The theory and research from previous chapters are used. The concept of a system is described. Explanation how system collects data from business systems and how visualises them is also discussed.
\item In \cref{ch:proof-of-concept} implementation of proof-of-concept is created. Architecture and used technologies are described. An example model of business processes is demonstrated.
\end{enumerate}

At Conclusions the results are summarized and discussed.



