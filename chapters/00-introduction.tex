Nowadays almost every company has more independent systems which help employees to do their job. For example one system for accountancy, another for warehouses and another one for managing orders or something similar. However how business grows, some kind of business analysis is required. Usually some person is instructed to analyse the business, how to make it better. The person collects all required data from every system, typically in some form of  ``Excel table'' and over it he does analysis. This approach at the beginning can be sufficient, however sooner or later, it will become heavy, uncomfortable and mainly not effective way how to do this kind of analysis. All required data are spread over all systems without any order. Some systems can have functions to export data in some suitable form, but from another systems they can export data only in some form of plain-text and so on.  At some point people find out that in their business ``reigns chaos''. Fortunately this ``chaos'' can be controlled with some well known approaches. Most well known term of how ``chaos' can be controlled is Business Process Management (BPM).

\section{Analysis within BPM}
Every task or set of tasks within the business can be formulated as \textit{business process}. For example task ``Order of new components for machine'' (which consists of many additional steps)  can be expressed as business process ``New machine components order''. The well-defined business processes are the first step to having better control over own business and also for further analysis. 

Next step is that these defined processes connect under one system with internal business applications. The ``one system'' is typically marked as BPM System (BPMS). Within BPMS the business processes are defined, modelled and executed. Modelling is typically done through Business Process Model and Notation (BPMN) which is a well-known method to model business processes. Execution means, that dependent internal systems within business communicates with BPMS and output from these systems is reflected in BPMS. This is the second important step to have a more precise analysis of own business.

If the company has some kind of BPMS, they have also well-defined data and aggregated them in one place. Usually, every BPMS offers functionalities to do analysis and monitoring over collected data. These functionalities can help people managing their business, more easily finding issues and potentially solving them. Monitoring is typically done through graphical overviews, which offer varied views on the business metrics. These metrics can be for example business efficiency, employee performance (how much the desired goals are fulfilled), financial efficiency and so on. Thanks to these systems people have the better understanding of their business and they can more easily focus on the critical places that can be improved.

\section{A New Approach}
While BPMN is well known and widely used, the new methodology was introduced by \textit{J.Dietz}~\cite{dietz-essence-2015}. This new methodology called DEMO is considered as successor of BPMN. DEMO is build on high-quality scientific foundations, which brings against BPMN (and similar methods) better modelling of business processes. Although DEMO is well defined methodology, BPMS based on DEMO does not (nearly) exists. The problem of BPMS based on DEMO is subject of many researches. There are first attempts to achieve a solution based on DEMO, which can design processes and execute them. However the monitoring and analysis are missing.

This thesis focuses on visualisation of business processes based on DEMO methodology. Within defined process the goal of thesis is to find a way:

\begin{itemize}
\item The data processing from internal systems and their connection within DEMO model. 
\item Offer data to users with appropriate graphical overviews.
\item Real-time visualisation of running processes.
\end{itemize}

\section{Motivation}
The main reason why I have focused on this topic comes from a school project. With our team we created chatbot application for pizza delivering system which was based on DEMO methodology. In fact we used \textit{DEMO Engine}, the BPMS based on DEMO developed by \textit{ForMetis} company\todo{citation needed}. Moreover, on this \textit{DEMO Engine} worked also my supervisor (and also the team leader of our team) \textit{M.Skotnica}~\cite{diploma-skotnica-2016}. With \textit{engine} we defined model within DEMO for ``imaginary'' pizzeria. Then we used this defined processes and connected them with the chatbot. However, this system had only ability to design, model and execute processes. The monitoring of collected data or visualisation of processes were missed. That's why I have chosen this topic.

\section{Structure}
The thesis is divided to four chapters:
\begin{itemize}
\item In \cref{ch:theoretical-foundations} underlying theories and terms are explained.
\item In \cref{ch:bpms} research of existing BPMS solutions follows. Then some specific solutions are investigated for analysis of own solution. Role of DEMO within BPMS is described. 
\item Then in \cref{ch:proposed-approach} the proposed approach is introduced. The theory and research from previous chapters are used. The concept of a system is described. Principle of real-time visualisation is more precisely explained. 
\item In \cref{ch:proof-of-concept} implementation of proof-of-concept is created. Architecture and used technologies are described. An example model of business processes is demonstrated.
\end{itemize}
In conclusion the results are compared with the goals of this thesis. 




