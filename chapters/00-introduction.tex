% In these days, there are many systems, that help companies to analyse and monitor their business processes and performance. These systems are well established and widely used. Besides of monitoring, better and more complex systems offer the whole bunch of functionalities, from the defining business process, running them to analysing and monitoring. However, these systems are built on top of notation BPMN, which is again, well known and widely used. The standardly used term for these systems is Business Process Management System (BPMS).

% This thesis relies on the theory of another approach - DEMO methodology, which is quite new and there are likely no systems build on this methodology. The thesis is not focused on every aspect of the existing BPMS solutions. The thesis is only focused on how processes can be monitored and displayed in graphical overviews. How processes can be modelled and executed is the different topic, which was researched for example by \textit{M. Skotnica}\cite{diploma-skotnica-2016}. 

% The motivation for focusing on this problem comes out from the school project, which was about creating chat-bot for pizza delivery system. This chatbot was relying on DEMO Engine, which is the system, that can design and run DEMO models. However, this system did not have the ability to display or visualise processes in some ``user-friendly'' graphical overview. These graphical overviews can help to better understand business processes, monitor performance and many more metrics, which can lead to more effective business.

% \subsection{Structure}
% The thesis is organized into main four chapters:
% \begin{enumerate}
% \item In \cref{ch:theoretical-foundations} underlying theories are introduced and described.
% \item In \cref{ch:bpms} research of existing BPMS solutions follows. Then some specific solutions are showed and how knowledge from research is used to propose a similar system based on DEMO.
% \item In \cref{ch:proposed-approach} proposed approach is explained. The theory and research from previous chapters are used. The concept of a system is described. Explanation how system collects data from business systems and how visualises them is also discussed.
% \item In \cref{ch:proof-of-concept} implementation of proof-of-concept is created. Architecture and used technologies are described. An example model of business processes is demonstrated.
% \end{enumerate}

% At Conclusions the results are summarized and discussed.

Nowadays systems for modelling a business processes are well known and used. Commonly these systems are called Business Process Management Systems(BPMS). They usually use Business Process Model and Notation (BPMN) for defining and modelling processes . Complex solutions offer whole bunch of functionalities from modelling business processes, their executing to monitoring and analysing. Every BPMS which offer full solution (designing, modelling,  executing, monitoring and optimizing) have functions which can help people to manage their business, more easily find issues and potentially solve them.  Monitoring is typically done through graphical overviews, which offer varied views on the business metrics. These metrics can be for example business efficiency, employee performance (how much the desired goals are fulfilled), financial efficiency and so on. Thanks to these systems people have better understanding of their business and they can more easily focus on the critical places that can be improved.

Against BPMN the new methodology was founded which is considered as (better) it's replacement. This new methodology is called DEMO and it was introduced by J. Dietz\todo{citation Dietz}. DEMO is build on high-quality scientific foundations, which brings against BPMN (and similar methods) better modelling of business processes.
Although DEMO is well defined methodology, BPMS based on DEMO does not (nearly) exists. The problem of BPMS based on DEMO is subject of many researches. There are first attempts to achieve solution based on DEMO, which can model and execute models within DEMO. However the monitoring and analysis are missing.

This thesis focuses on visualisation of business processes based on DEMO methodology. Within defined process the goal of thesis is to find a way:

\begin{itemize}
\item The data processing from internal systems and their connection within DEMO model. 
\item Offer data to users with appropriate graphical overviews.
\item Real-time visualisation of running processes.
\end{itemize}

\subsection{Motivation}

\todo{todo}

\subsection{Structure}
The thesis is divided to four chapters:
\begin{itemize}
\item In \cref{ch:theoretical-foundations} underlying theories and terms are explained.
\item In \cref{ch:bpms} research of existing BPMS solutions follows. Then some specific solutions are investigated for analysis of own solution. Role of DEMO within BPMS is described. 
\item Then in \cref{ch:proposed-approach} the proposed approach is introduced. The theory and research from previous chapters are used. The concept of a system is described. Principle of real-time visualisation is more precisely explained. 
\item In \cref{ch:proof-of-concept} implementation of proof-of-concept is created. Architecture and used technologies are described. An example model of business processes is demonstrated.
\end{itemize}
At conclusions the results are compared with thesis goals.




