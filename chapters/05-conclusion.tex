In the introduction and first two chapter were introduced theoretical foundations and showed existing BPM solutions which are commercially used. The goal of thesis was mainly propose a way how visualisation of processes defined within DEMO can be done. A few approaches were inntroduced. Than followed explanation why chosen approach is the best for the visualisation. 

The possibility of real usage was proofed through mobile application. The goal was to provide a simple to use and to understand approach, which seems to be accomplished. The visualisation is highly inspired from widely used Gantt charts, which are popular mainly in task-management. Also the concept of widgets within dashboard was described. These widgets provides user high valuable overview of collected data which can really help to improve their business. 

\subsection{Future steps}
A visualisation itself could be defined more universally and could provide a detailed approach how any enterprise system can be mapped to defined DEMO models. As has been noted, a few existing solution based on DEMO exists. The really next step could be to include this visualisation  approach into existing BPM solutions based on DEMO. Thanks to this step a complete BPM solution could exists.